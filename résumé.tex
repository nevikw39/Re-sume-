\documentclass[12pt, a4paper]{article}

\title{R\'{e}sum\'{e}}
\author{\textsc{Chun-Mu, Weng}\footnote{\url{https://github.com/nevikw39/}}\\{\footnotesize\href{mailto:nevikw39@m110.nthu.edu.tw}{\ttfamily nevikw39@m110.nthu.edu.tw}}}
\date{\today}

\usepackage{float}
\usepackage{amsmath}
\usepackage{amssymb}
\usepackage{caption}
\usepackage{subcaption}
\usepackage{tikz}
\usepackage{pgfplots}
\usepackage{listings}
\usepackage{hyperref}
\usepackage{booktabs}
\usepackage{longtable}

\usepackage[margin=2cm]{geometry}

\begin{document}
\maketitle

I'm an enthusiastic competitive programmer who is eager to seek an opportunity to put all my strength into resolving real-life problems.

\section{Academic Performance}

%Currently I'm a sophomore major in \textsf{Computer Science} in National Tsing Hua University (Taiwan), expected to graduate in Jun., 2025.
%
%My GPA in the first semester, \emph{taking 28 credits} in Fall 2021, was \textbf{4.21}, ranking \textbf{2\textsuperscript{nd}}/45, \textbf{6\textsuperscript{th}}/130 at class and department respectively. Overall, my GPA in the frosh year was \textbf{4.09}.

\begin{itemize}
\item Currently a sophomore at National Tsing Hua University (Taiwan), majoring in \textsf{Computer Science} (expected to graduate in Jun., 2025)
\item GPA of \textbf{4.21} Fall 2021, my 1\textsuperscript{st} semester \emph{with 28 credits}, ranking \textbf{2\textsuperscript{nd}}/45 in class and \textbf{6\textsuperscript{th}}/130 in department\\
Overall GPA of \textbf{4.09} in freshman year
\end{itemize}

\section{Leaderships \& Awards}

\subsection{Programming Contests}

%My voyage of competitive programming rooted in high school. I continue to compete in several prominent and prestigious programming contests ever since I was a college freshman. Among these contests, not only did I perfect and elaborate my programming proficiency but also learnt the authentic true value of \textbf{teamwork}.

I started competitive programming since high school. Here are some prominent and prestigious contest I participated in college:

\begin{itemize}
\item 2021 National Collegiate Programming Contest Final (Taiwan), Team \emph{Dkjistra}, \textbf{Honorable Mention}, ranked 28\textsuperscript{th}.
\item 2021 Interational Collegiate Programming Contest Asia Taipei Regional, Team \emph{Dkjistra}, \textbf{Silver Award}, ranked 30\textsuperscript{th}.
\item 2022 National Collegiate Programming Contest Final (Taiwan), Team \emph{DebugCat Capoo}, \textbf{Honorable Mention}, ranked 23\textsuperscript{th}.
\item 2022 Interational Collegiate Programming Contest Asia Taoyuan Regional, Team \emph{DebugCat Capoo}, \textbf{Bronze Award}, ranked 39\textsuperscript{th}.
\end{itemize}

\subsection{High-Performance Computing Competitions}

%HPC competitions, or cluster competitions, are a sort of competitions with tasks required participants to solve with the cluster supercomputers. Advised by Prof. Jerry Chou and some experienced senior students, 4 sophomores and 2 juniors teamed up and got along with HPC from scratch. Since I had more background knowledge and learnt fast, I was elected the team \textbf{leader}. I took up the responsibilities and spared no effort to maintain our team.

\begin{itemize}
\item 4 sophomores \& 2 juniors teamed up
\item Elected as the leader due to the background knowledge \& ability to learn quickly
\item Took on responsibilities and maintain the team
\end{itemize}

\subsubsection{2022 HiPAC (Taiwan)}

%We registered the first High Performance Application Competition organised by NCHC Taiwan. Our proposal was to profile and optimise the efficiency of calculating future green energy mechanisms \textbf{parallellly} across 16 nodes of the cluster supercomputer \emph{Taiwania 3} by the electronic-structure software Quantum \textsc{Espresso} and we won the \textbf{3\textsuperscript{rd} place}.

\begin{itemize}
\item 1\textsuperscript{st} High Performance Application Competition by NCHC Taiwan
\item Profiled and optimized the efficiency of calculating future green energy mechanisms using Quantum \textsc{Espresso} software across 16 nodes of the cluster supercomputer \emph{Taiwania 3}
\item Won \textbf{3\textsuperscript{rd} place} in the competition
\end{itemize}

\subsubsection{2022 APAC HPC-AI Competition}

%The six-month long Asia-Pacific High-Performance Computing AI Competition\footnote{\url{https://www.hpcadvisorycouncil.com/2022_APAC_HPC_AI\%20Competition\%20Result_Announcement_PR.pdf}}, co-organised by HPC-AI Advisory Council, NSCC Singapore and NCI Australia and involved 25 teams from 12 countries, included tasks focusing on three critical issues that leverage the power of HPC and AI to develop solutions to human health and environmental sustainability.
%
%We improve the throughput of communications between 16 \textsf{NVIDIA V100} across 4 GPU nodes of the cluster supercomputer \emph{Gadi} about 2.7 times via Active Message scheme of \textsf{UCX} Rendezvous protocol and some optimisation. The accuracy of deep-learning-based DNA sequence fast decoding was increased by adopting suitable model and the training time was reduced by running on \textsf{NVIDIA A100}.

\renewcommand\UrlFont{\ttfamily\scriptsize}

\begin{itemize}
\item 5\textsuperscript{th} Asia-Pacific High-Performance Computing \& AI Competitionn\footnote{\url{https://www.hpcadvisorycouncil.com/2022_APAC_HPC_AI\%20Competition\%20Result_Announcement_PR.pdf}}, co-organised by HPC-AI Advisory Council, NSCC Sg. \& NCI Au. and involved 25 teams from 12 countries
\item Resolved tasks on 3 critical issues that leverage HPC \& AI to develop solutions to human health \& environmental sustainability
\item Improved throughput between 16 \textsf{NVIDIA V100} on 4 nodes of the supercomputer \emph{Gadi} by 2.7 times via Active Message scheme of \textsf{UCX} Rendezvous protocol \& optimization
\item Increased accuracy of DL-based DNA sequence decoding and reduced training time using \textsf{NVIDIA A100}
\end{itemize}

Our team were granted the following honor:

\begin{itemize}
\item \textbf{1\textsuperscript{st} place} crowned the overall champion trophy
\item Best Big Data Analytics Performance Award
\item A reserved slot at 2023 ISC in Germany
\end{itemize}

\section{Experiences}

\subsection{Teaching Assistant}

%I served as the TA for \textsc{Introduction to Programming \MakeUppercase{\romannumeral2}} in Spring 2022, when I was a freshman then. I'm supposed to be the TA again for this course in Spring 2023.
%
%My main task was to design the online judge problems, which helped me to \textbf{internalize} and \textbf{reinforce} my expertise and mastery in \textbf{data structures} and \textbf{algorithms}.

\begin{itemize}
\item Served as TA for \textsc{Introduction to Programming \MakeUppercase{\romannumeral2}} in Spring 2022 as a freshman\\
Expected to be TA again in Spring 2023
\item Main task was to design online judge problems, which improved expertise \& mastery in \textbf{data structures} \& \textbf{algorithms}
\end{itemize}

\subsection{Lecturer of Study Group}

%There is a study group aimed at providing basic yet practical tutorials for students, especially freshmen, called \textsf{CSST} \textit{(Computer Science STudents)} established by senior students in our department.
%
%When I became sophomore, I joined in Teaching Division of \textsf{Computer Science} Student Association, contributing to the preparation of \textsf{CSST} this year. Furthermore, I also lectured two tutorials about \textsf{Linux} terminal environments and \LaTeX.

\begin{itemize}
\item Joined Teaching Div. of CS Student Assoc. in sophomore year\\
Contributed to \textsf{CSST} \textit{(CS STudents)}, a study group providing basic tutorials to students
\item Lectured two ones on \textsf{Linux} terminal environments and \textsf{Markdown} \& \LaTeX
\end{itemize}

\section{Skills}

\begin{description}
\item[Languages] \texttt{C/C++}, \texttt{Python} (main ones), \texttt{JavaScript}, \texttt{Verilog} (less-familiar), \texttt{Go}, \texttt{Lua}, \texttt{Ruby}, \texttt{Rust}, \texttt{Java}
\item[Tools \& Libraries] \textsf{Git}, \textsf{Unix}-like OSs \& shells, \textsf{Docker}, \textsf{MPI}, \textsf{OpenMP}, \textsf{CUDA}
\end{description}

\end{document}
